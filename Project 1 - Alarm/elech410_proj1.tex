\documentclass[10pt,a4paper]{article}
\usepackage[utf8x]{inputenc}
\usepackage[T1]{fontenc}
%\usepackage{stringenc} % for grffile
\usepackage{ucs}
\usepackage{amsthm} %numéroter les questions
\usepackage[english]{babel}
\usepackage{datetime}
\usepackage{xspace} % typographie IN
\usepackage{hyperref}% hyperliens
\usepackage[all]{hypcap} %lien pointe en haut des figures
\usepackage[english]{varioref} %voir x p y
\usepackage{fancyhdr}% en têtes
%\input cyracc.def
\usepackage[]{graphicx} %include pictures
%\usepackage[encoding,inputencoding=utf8,filenameencoding=utf8]{grffile}
%\usepackage[extendedchars,inputencoding=latin1,filenameencoding=latin1]{grffile}
\usepackage[siunitx ]{circuitikz}
\usepackage{gnuplottex}
\usepackage{ifthen}
\graphicspath{{./figures/}}
%\usepackage{array}
\usepackage{amsmath}
\usepackage[]{xcolor}
\usepackage{tikz}
\usepackage{tikz-timing}
\usetikzlibrary{scopes}
\usetikzlibrary{backgrounds}
\usepackage{listings}
\usepackage{enumitem}
\usepackage[top=1 in, bottom=1 in, left=1.3 in, right=1 in]{geometry} % Yeah, that's bad to play with margins
\usepackage[]{pdfpages}
\usepackage{pdflscape}
\usepackage[]{attachfile}
%\usepackage{colortbl}
%\usepackage{multirow}
\usepackage{booktabs}
\usepackage{makecell}
\usepackage[ ]{subfig}
%\usepackage{rotating}
\usepackage{upgreek}

\newdateformat{mydate}{2017--2018}%hack pour remplacer \THEYEAR

%cyr
%\newcommand\textcyr[1]{{\fontencoding{OT2}\fontfamily{wncyr}\selectfont #1}}
\renewcommand{\marginpar}[1]{} %surcharge pour /marginpar

\newboolean{corrige}
%\setboolean{corrige}{true}%corrigé
\setboolean{corrige}{false}% pas de corrigé

\newboolean{annexes}
%\setboolean{annexes}{true}%annexes
\setboolean{annexes}{false}% pas de annexes

\newboolean{mos}
%\setboolean{mos}{true}%annexes
\setboolean{mos}{false}% pas de annexes

\usepackage{aeguill} %guillemets

%% fancy header & foot
\pagestyle{fancy}
\lhead{[ELEC-H-410] Real-Time Computer Systems Project: Distributed alarm}
\rhead{\mydate\today\\ page \thepage}
\chead{\ifthenelse{\boolean{corrige}}{Corrigé}{}}
\cfoot{}
%%

\pdfinfo{
/Author (ULB -- BEAMS)
/Title (Project ELEC-H-410, uCOS-II)
/ModDate (D:\pdfdate)
}

\hypersetup{
pdftitle={Project [ELEC-H-410] Real-Time Computer Systems},
pdfauthor={ULB - BEAMS  },
pdfsubject={uCOS-II}
}

\theoremstyle{definition}% questions pas en italique
\newtheorem{E}{\color{blue}Exercise}[] % numéroter les questions [section] ou non []

\newcommand{\reponse}[1]{% pour intégrer une réponse : \reponse{texte} : sera inclus si \boolean{corrige}
	\ifthenelse {\boolean{corrige}} {\paragraph{Réponse :} #1} {}
 }

\newcommand{\addcontentslinenono}[4]{\addtocontents{#1}{\protect\contentsline{#2}{#3}{#4}{}}}

\newcommand{\on}[1]{\operatorname{#1}}

\newcommand{\reg}[1]{\texttt{reg#1}}

\newcommand{\uCOSII}{$\upmu$C/OS-II}

\newcommand{\kw}[1]{\texttt{#1}}

\setlength{\parskip}{1ex plus .5ex minus .5ex} % espacement entre paragraphes
\setlength{\parindent}{0 ex plus 0ex minus 0 ex} % retrait en début de §

\def\labelitemi{--}
\setlist{parsep=0pt,itemsep=0pt,style=standard,leftmargin=\parindent, align=left} % pas d'espace prohibitif entre les items
\setlist{nolistsep}

\newcolumntype{C}[1]{>{\centering\let\newline\\\arraybackslash\hspace{0pt}}m{#1}}

%\setlength{\tabcolsep}{0pt} %no extra space in cells to keep constant tabular width

\date{\vspace{-1cm}\mydate\today}
\title{\vspace{-2cm} Project\\ Real-Time Computer Systems [ELEC-H-410]\\ \textbf{Design a real-time distributed alarm}\ifthenelse{\boolean{corrige}}{~\\Corrigé}{}}

%\author{\vspace{-1cm}}%\textsc{Yannick Allard}}


\lstdefinestyle{customasm}{
 % belowcaptionskip=1\baselineskip,
 % frame=L,
 % xleftmargin=\parindent,
  language=[x86masm]Assembler,
  basicstyle=\footnotesize\ttfamily,
  commentstyle=\itshape\color{purple!40!black},
  comment=[l]//,
}
\lstdefinestyle{customc}{
  belowcaptionskip=1\baselineskip,
  breaklines=true,
  frame=L,
  xleftmargin=\parindent,
  language=C,
  showstringspaces=false,
  basicstyle=\footnotesize\ttfamily,
  keywordstyle=\bfseries\color{green!40!black},
  commentstyle=\itshape\color{purple!40!black},
  identifierstyle=\color{blue},
  stringstyle=\color{orange},
}
\lstset{escapechar=@,style=customc}

\begin{document}

% Introduce a new counter for counting the nodes needed for circling
\newcounter{nodecount}
% Command for making a new node and naming it according to the nodecount counter
\newcommand\tabnode[1]{\addtocounter{nodecount}{1} \tikz \node (\arabic{nodecount}) {#1};}

% Some options common to all the nodes and paths
\tikzstyle{every picture}+=[remember picture,baseline]
\tikzstyle{every node}+=[inner sep=0pt,anchor=base]
\tikzstyle{every path}+=[thick, rounded corners]

% for tikz pict


\maketitle
\section{Purpose}

On the basis of the previous labs you will now create a distributed alarm system working on the CAN network.



\section{Useful documents are stored on the network share:}
 \verb!\\labo\ELEC-H-410\Useful Documents\!%\marginpar{OK update this}
\texttt{\begin{itemize}
\item dsPIC30F-33F Programmer's Reference.pdf
\item dsPIC33 Data Sheet.pdf section 21
\item Introduction to MPLAB.pdf
\item Explorer 16 User Guide 51589a.pdf
\item MPLAB C30 C Compiler User's guide.pdf
\item uCOSII\_RefMan.pdf
\item Enhanced Controller Area Network.pdf
\item Introduction to language C for microcontrollers.pdf
\item Table of ASCII codes.
\item Troubleshooting \uCOSII 
\end{itemize}}

%*
%*
%Introduction à MPLAB.pdf*
%*
%La Carte Explorer16.pdf*
%*
%*


\section{Specifications}
In this distributed alarm system, all nodes are equal (i.e. there is no master node), can work
autonomously but also exchange information through CAN. The application of the node should run under \uCOSII.\\

Each node of the alarm system will have to carry the following functionalities out:
\begin{itemize}
\item Monitor the presence of intruders. %You will have at your disposal an infra-red barrier that
%you can connect to a digital input of the microcontroller.
The sensor will be a push button connected to a digital input of the microcontroller.

\item Arm/disarm the system with a secret code typed on the keyboard. For the moment, you can
use the value B169. Functions to change this code are optional.
\item If an intruder is detected while the system is armed, leave the possibility to disarm the
system using the same code during the first 30s.
\item If the system has not been disarmed within 30s, the alarm should start. You will get a
buzzer or a flashing light that you can connect to an output port.
\end{itemize}

\vspace{0.5cm}

In a distributed system, each node must cooperate with the other members of the network:
\begin{itemize}
\item A node signals its presence when it appears on the network and repeat this message
periodically at least every 5 seconds. This ``heartbeat" enables to detect the disappearance
of the node (failure or sabotage).
\item If a node detects an intrusion, it must signal it so that all other nodes pass in the ``alert"
state, even if didn't detect anybody.
\item (Dis)arming a node should be propagated to all other nodes; there is no obligation to disarm
at the node which has detected the intrusion.
\item Think about any way to attack your alarm and try to reduce possibilities of success in doing that.
\end{itemize}

\newpage

To facilitate the management of the communication, we suggest that you use the following
protocol:
\begin{itemize}
%\item Each node sends only 1 object (CAN\_ID) between 1 and 254. You need different types of
%messages, the type will be indicated by the data (see Table \ref{Table of messages}).\\
%OR
%\item Each node sends only 1 data (DATA[0]). You need different types of messages, the type will be indicated by 1 object (CAN\_ID) between 1 and 254 that will also define the priority of the message (see Table \ref{Table of messages}).
\item \texttt{Message\_ID} will be fixed during the first brainstorm and is linked to the priority of the messages on the network. Nodes must specify their \texttt{Node\_ID} in the data, one byte, value 0 to 255.
\item The alarm system is limited to 10 nodes whose  \texttt{Node\_ID} is unknown a priori.
\item Then a node boots, it has to find a free identifier, then signal its presence by broadcasting
it regularly.
\end{itemize}

\begin{table} [ht]   %h pour le placement voir vers p110
					\begin{center}
						%Le tableau en tant que tel
						\begin{tabular}{|l|c|c|}
								\hline
								\begin{bf}Message type\end{bf}&\begin{bf}Message\_ID\end{bf}&\begin{bf}Data\end{bf}\\
								\hline
								Heartbeat& 0 &node ID (one byte)\\
								\hline
								Intrusion detected& 1 &node ID\\
								\hline
								Node has been armed& 4 &node ID\\
								\hline
								Node has been disarmed& 2 &node ID\\
								\hline
								Alarm started& 8 &node ID\\
								\hline
								New password & 24 &node ID + password (4 characters)\\
								\hline
						\end{tabular}
					\end{center}
					\caption{Table of messages}
					\label{Table of messages}
					\end{table}


\section{Operating mode}
\textbf{\color{red}The project will be done in group of 3 or 4.}
\subsection{Timetable}
Five sessions are dedicated to the project:
\begin{itemize}
\item During the \emph{first project session}, you have to brainstorm about the specifications of a distributed alarm. Don't hesitate to use timing diagrams, block diagrams, state machines  or other operating/functional schemes to share your ideas. You must \textbf{NOT} code anything during this session.
\item During the \emph{next three sessions}, you will use your knowledge and the result of your brainstorming to implement your node of the distributed alarm system. 
\item Your project will be evaluated during the \emph{last session}.
\end{itemize}


\subsection{Evaluation}
The project will we evaluated on the basis of a short presentation of your work and your answers to questions.\\
\textsc{F. Quitin} and \textsc{S. Mezhoud} will spend several minutes in each group during the last session to check the effective functioning of your alarm, to assess the quality of your code and to ask you a few questions.

\begin{flushright}
\section*{Good luck!}
\end{flushright}




\end{document}
