\section*{Find the code !}

To make sure you understood how CAN communication works, here is a final task.

Your goal is to retrieve a code from a node connected to the network (the \textit{master} node).
The difficulty is that the master node requires a special procedure in order to deliver the code.

The master node sends sporadic messages containing a single byte (\textit{token}).
If you want to know the code you should respond by encoding the token and sending it within $ 500~ms $.
The master will respond with the code in ASCII that you can directly print it on your LCD.
\newcommand{\bloodymess}[7][0]{
  \stepcounter{seqlevel}
  \path
  (#2)+(0,-\theseqlevel*\unitfactor-0.7*\unitfactor) node (mess from) {};
  \addtocounter{seqlevel}{#1}
  \path
  (#4)+(0,-\theseqlevel*\unitfactor-0.7*\unitfactor) node (mess to) {};
  \draw[->,>=angle 60] (mess from) -- (mess to) node[midway, above]
  {#3};

  \if R#5
    \node (#3 from) at (mess from) {\llap{#6~}};
    \node (#3 to) at (mess to) {\rlap{~#7}};
  \else\if L#5
         \node (#3 from) at (mess from) {\rlap{~#6}};
         \node (#3 to) at (mess to) {\llap{#7~}};
       \else
         \node (#3 from) at (mess from) {#6};
         \node (#3 to) at (mess to) {#7};
       \fi
  \fi
}
\begin{figure}[h]
    \centering
    \begin{sequencediagram}
        \newinst{c}{Master}
        \newinst[6]{s}{Node}

        \bloodymess[1]{c}{Token on 0x020}{s}{R}{Random start}{}
        \bloodymess[1]{s}{Encrypted token on 0x03X}{c}{L}{}{$ > 500~ms $}
        \bloodymess[1]{c}{Code in ASCII on 0x03X}{s}{R}{}{}
    \end{sequencediagram}
    \caption{}
\end{figure}

Open the project \kw{escape\_game}.

To avoid any overlaps between groups, every groups will listen and transmit messages with ID equal to their table plus 30 in hexadecimal e.g. table 5 will listen and transmit messages with ID 0x035.
The initial token will be sent on the 0x020 ID.

To decode the token we provide you the function \kw{decode(token)}, it depends on your table number so you need to change it. 
