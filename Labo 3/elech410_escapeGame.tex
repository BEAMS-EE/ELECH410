\documentclass[10pt,a4paper]{article}
\usepackage[utf8x]{inputenc}
\usepackage[T1]{fontenc}
%\usepackage{stringenc} % for grffile
\usepackage{ucs}
\usepackage{amsthm} %numéroter les questions
\usepackage[english]{babel}
\usepackage{datetime}
\usepackage{xspace} % typographie IN
\usepackage{hyperref}% hyperliens
\usepackage[all]{hypcap} %lien pointe en haut des figures
\usepackage[english]{varioref} %voir x p y
\usepackage{fancyhdr}% en têtes
%\input cyracc.def
\usepackage[]{graphicx} %include pictures
%\usepackage[encoding,inputencoding=utf8,filenameencoding=utf8]{grffile}
%\usepackage[extendedchars,inputencoding=latin1,filenameencoding=latin1]{grffile}
\usepackage[siunitx ]{circuitikz}
\usepackage{gnuplottex}
\usepackage{ifthen}
\graphicspath{{./figures/}}
%\usepackage{array}
\usepackage{amsmath}
\usepackage[]{xcolor}
\usepackage{tikz}
\usepackage{tikz-timing}
\usetikzlibrary{scopes}
\usetikzlibrary{backgrounds}
\usepackage{listings}
\usepackage{enumitem}
\usepackage[top=1 in, bottom=1 in, left=1.3 in, right=1 in]{geometry} % Yeah, that's bad to play with margins
\usepackage[]{pdfpages}
\usepackage{pdflscape}
\usepackage[]{attachfile}
%\usepackage{colortbl}
%\usepackage{multirow}
\usepackage{booktabs}
\usepackage{makecell}
\usepackage[ ]{subfig}
%\usepackage{rotating}
\usepackage{upgreek}

\usepackage{float}
\usepackage{tikz}
\usetikzlibrary{positioning, fit, calc, shapes, arrows}
\usepackage[underline=false]{pgf-umlsd}

\newdateformat{mydate}{2018--2019}%hack pour remplacer \THEYEAR

%cyr
%\newcommand\textcyr[1]{{\fontencoding{OT2}\fontfamily{wncyr}\selectfont #1}}
\renewcommand{\marginpar}[1]{} %surcharge pour /marginpar

\newboolean{corrige}
%\setboolean{corrige}{true}%corrigé
\setboolean{corrige}{false}% pas de corrigé

\newboolean{annexes}
%\setboolean{annexes}{true}%annexes
\setboolean{annexes}{false}% pas de annexes

\newboolean{mos}
%\setboolean{mos}{true}%annexes
\setboolean{mos}{false}% pas de annexes

\usepackage{aeguill} %guillemets

%% fancy header & foot
\pagestyle{fancy}
\lhead{[ELEC-H-410] Real-Time systems Lab n° 3: \rtos}
\rhead{\mydate\today\\ page \thepage}
\chead{\ifthenelse{\boolean{corrige}}{Corrigé}{}}
\cfoot{}
%%

\pdfinfo{
    /Author (ULB -- BEAMS)
    /Title (Lab 3 ELEC-H-410)
    /ModDate (D:\pdfdate)
}

\hypersetup{
    pdftitle={Lab n° 3 [ELEC-H-410] Real-Time systems},
    pdfauthor={ULB - BEAMS},
    pdfsubject={FreeRTOS}
}

\theoremstyle{definition}% questions pas en italique
\newtheorem{E}{\color{blue}Exercise}[] % numéroter les questions [section] ou non []

\newcommand{\reponse}[1]{% pour intégrer une réponse : \reponse{texte} : sera inclus si \boolean{corrige}
	\ifthenelse {\boolean{corrige}} {\paragraph{Answer :} #1} {}
 }

\newcommand{\addcontentslinenono}[4]{\addtocontents{#1}{\protect\contentsline{#2}{#3}{#4}{}}}

\newcommand{\on}[1]{\operatorname{#1}}

\newcommand{\reg}[1]{\texttt{reg#1}}

%TODO Remove that
\newcommand{\uCOSII}{$\upmu$C/OS-II}
\newcommand{\rtos}{FreeRTOS}

\newcommand{\kw}[1]{\texttt{#1}}

\setlength{\parskip}{1ex plus .5ex minus .5ex} % espacement entre paragraphes
\setlength{\parindent}{0 ex plus 0ex minus 0 ex} % retrait en début de §

\def\labelitemi{--}
\setlist{parsep=0pt,itemsep=0pt,style=standard,leftmargin=\parindent, align=left} % pas d'espace prohibitif entre les items
\setlist{nolistsep}

\newcolumntype{C}[1]{>{\centering\let\newline\\\arraybackslash\hspace{0pt}}m{#1}}

%\setlength{\tabcolsep}{0pt} %no extra space in cells to keep constant tabular width

\date{\vspace{-1cm}\mydate\today}
\title{\vspace{-2cm} Lab n° 3\\ Real-Time systems [ELEC-H-410]\\ CAN network overview\ifthenelse{\boolean{corrige}}{~\\Corrigé}{}}

%\author{\vspace{-1cm}}%\textsc{Yannick Allard}}


\lstdefinestyle{customasm}{
    % belowcaptionskip=1\baselineskip,
    % frame=L,
    % xleftmargin=\parindent,
    language=[x86masm]Assembler,
    basicstyle=\footnotesize\ttfamily,
    commentstyle=\itshape\color{purple!40!black},
    comment=[l]//,
}
\lstdefinestyle{customc}{
    belowcaptionskip=1\baselineskip,
    breaklines=true,
    frame=L,
    xleftmargin=\parindent,
    language=C,
    showstringspaces=false,
    basicstyle=\footnotesize\ttfamily,
    keywordstyle=\bfseries\color{green!40!black},
    commentstyle=\itshape\color{purple!40!black},
    identifierstyle=\color{blue},
    stringstyle=\color{orange},
}
\lstset{escapechar=@,style=customc}

\begin{document}

    % Introduce a new counter for counting the nodes needed for circling
    \newcounter{nodecount}
    % Command for making a new node and naming it according to the nodecount counter
    \newcommand\tabnode[1]{\addtocounter{nodecount}{1} \tikz \node (\arabic{nodecount}) {#1};}
    
    % Some options common to all the nodes and paths
    \tikzstyle{every picture}+=[remember picture,baseline]
    \tikzstyle{every node}+=[inner sep=0pt,anchor=base]
    \tikzstyle{every path}+=[thick, rounded corners]
    
    % for tikz pict

    \maketitle
    \section*{Find the code !}

To make sure you understood how CAN communication works, here is a final task.

Your goal is to retrieve a code from a node connected to the network (the \textit{master} node).
The difficulty is that the master node requires a special procedure in order to deliver the code.

The master node sends sporadic messages containing a single byte (\textit{token}).
If you want to know the code you should respond by encoding the token and sending it within $ 500~ms $.
The master will respond with the code in ASCII that you can directly print it on your LCD.
\newcommand{\bloodymess}[7][0]{
  \stepcounter{seqlevel}
  \path
  (#2)+(0,-\theseqlevel*\unitfactor-0.7*\unitfactor) node (mess from) {};
  \addtocounter{seqlevel}{#1}
  \path
  (#4)+(0,-\theseqlevel*\unitfactor-0.7*\unitfactor) node (mess to) {};
  \draw[->,>=angle 60] (mess from) -- (mess to) node[midway, above]
  {#3};

  \if R#5
    \node (#3 from) at (mess from) {\llap{#6~}};
    \node (#3 to) at (mess to) {\rlap{~#7}};
  \else\if L#5
         \node (#3 from) at (mess from) {\rlap{~#6}};
         \node (#3 to) at (mess to) {\llap{#7~}};
       \else
         \node (#3 from) at (mess from) {#6};
         \node (#3 to) at (mess to) {#7};
       \fi
  \fi
}
\begin{figure}[h]
    \centering
    \begin{sequencediagram}
        \newinst{c}{Master}
        \newinst[6]{s}{Node}

        \bloodymess[1]{c}{Token on 0x020}{s}{R}{Random start}{}
        \bloodymess[1]{s}{Encrypted token on 0x03X}{c}{L}{}{$ > 500~ms $}
        \bloodymess[1]{c}{Code in ASCII on 0x03X}{s}{R}{}{}
    \end{sequencediagram}
    \caption{}
\end{figure}

Open the project \kw{escape\_game}.

To avoid any overlaps between groups, every groups will listen and transmit messages with ID equal to their table plus 30 in hexadecimal e.g. table 5 will listen and transmit messages with ID 0x035.
The initial token will be sent on the 0x020 ID.

To decode the token we provide you the function \kw{decode(token)}, it depends on your table number so you need to change it. 


    \vfill
    \footnotesize{
        Found an error? Let us know: \url{https://github.com/BEAMS-EE/ELECH410/issues}
    }

\end{document}
