\documentclass[10pt,a4paper]{article}
\usepackage[utf8x]{inputenc}
\usepackage[T1]{fontenc}
%\usepackage{stringenc} % for grffile
\usepackage{ucs}
\usepackage{amsthm} %numéroter les questions
\usepackage[english]{babel}
\usepackage{datetime}
\usepackage{xspace} % typographie IN
\usepackage{hyperref}% hyperliens
\usepackage[all]{hypcap} %lien pointe en haut des figures
\usepackage[english]{varioref} %voir x p y
\usepackage{fancyhdr}% en têtes
%\input cyracc.def
\usepackage[]{graphicx} %include pictures
%\usepackage[encoding,inputencoding=utf8,filenameencoding=utf8]{grffile}
%\usepackage[extendedchars,inputencoding=latin1,filenameencoding=latin1]{grffile}
\usepackage[siunitx ]{circuitikz}
\usepackage{gnuplottex}
\usepackage{ifthen}
\graphicspath{{./figures/}}
%\usepackage{array}
\usepackage{amsmath}
\usepackage[]{xcolor}
\usepackage{tikz}
\usepackage{tikz-timing}
\usetikzlibrary{scopes}
\usetikzlibrary{backgrounds}
\usepackage{listings}
\usepackage{enumitem}
\usepackage[top=1 in, bottom=1 in, left=1.3 in, right=1 in]{geometry} % Yeah, that's bad to play with margins
\usepackage[]{pdfpages}
\usepackage{pdflscape}
\usepackage[]{attachfile}
%\usepackage{colortbl}
%\usepackage{multirow}
\usepackage{booktabs}
\usepackage{makecell}
\usepackage[ ]{subfig}
%\usepackage{rotating}
\usepackage{upgreek}

\newdateformat{mydate}{2019--2020}%hack pour remplacer \THEYEAR

%cyr
%\newcommand\textcyr[1]{{\fontencoding{OT2}\fontfamily{wncyr}\selectfont #1}}
\renewcommand{\marginpar}[1]{} %surcharge pour /marginpar

\newboolean{corrige}
%\setboolean{corrige}{true}%corrigé
\setboolean{corrige}{false}% pas de corrigé

\newboolean{annexes}
%\setboolean{annexes}{true}%annexes
\setboolean{annexes}{false}% pas de annexes

\newboolean{mos}
%\setboolean{mos}{true}%annexes
\setboolean{mos}{false}% pas de annexes

\usepackage{aeguill} %guillemets

%% fancy header & foot
\pagestyle{fancy}
\lhead{[ELEC-H-410] Real-Time systems Project}
\rhead{\mydate\today\\ page \thepage}
\chead{\ifthenelse{\boolean{corrige}}{Corrigé}{}}
\cfoot{}
%%

\pdfinfo{
    /Author (ULB -- BEAMS)
    /Title (Lab 3 ELEC-H-410)
    /ModDate (D:\pdfdate)
}

\hypersetup{
    pdftitle={Project [ELEC-H-410] Real-Time systems},
    pdfauthor={ULB - BEAMS},
    pdfsubject={FreeRTOS}
}

\theoremstyle{definition}% questions pas en italique
\newtheorem{E}{\color{blue}Exercise}[] % numéroter les questions [section] ou non []

\newcommand{\reponse}[1]{% pour intégrer une réponse : \reponse{texte} : sera inclus si \boolean{corrige}
	\ifthenelse {\boolean{corrige}} {\paragraph{Answer :} #1} {}
 }

\newcommand{\addcontentslinenono}[4]{\addtocontents{#1}{\protect\contentsline{#2}{#3}{#4}{}}}

\newcommand{\on}[1]{\operatorname{#1}}

\newcommand{\reg}[1]{\texttt{reg#1}}

\newcommand{\rtos}{FreeRTOS}

\newcommand{\kw}[1]{\texttt{#1}}

\setlength{\parskip}{1ex plus .5ex minus .5ex} % espacement entre paragraphes
\setlength{\parindent}{0 ex plus 0ex minus 0 ex} % retrait en début de §

\def\labelitemi{--}
\setlist{parsep=0pt,itemsep=0pt,style=standard,leftmargin=\parindent, align=left} % pas d'espace prohibitif entre les items
\setlist{nolistsep}

\newcolumntype{C}[1]{>{\centering\let\newline\\\arraybackslash\hspace{0pt}}m{#1}}

%\setlength{\tabcolsep}{0pt} %no extra space in cells to keep constant tabular width

\date{\vspace{-1cm}\mydate\today}
\title{\vspace{-2cm} Project\\ Real-Time systems [ELEC-H-410]\\ \textbf{Pandemic, the RTOS game} \ifthenelse{\boolean{corrige}}{~\\Corrigé}{}}

%\author{\vspace{-1cm}}%\textsc{Yannick Allard}}


\lstdefinestyle{customasm}{
    % belowcaptionskip=1\baselineskip,
    % frame=L,
    % xleftmargin=\parindent,
    language=[x86masm]Assembler,
    basicstyle=\footnotesize\ttfamily,
    commentstyle=\itshape\color{purple!40!black},
    comment=[l]//,
}
\lstdefinestyle{customc}{
    belowcaptionskip=1\baselineskip,
    breaklines=true,
    frame=L,
    xleftmargin=\parindent,
    language=C,
    showstringspaces=false,
    basicstyle=\footnotesize\ttfamily,
    keywordstyle=\bfseries\color{green!40!black},
    commentstyle=\itshape\color{purple!40!black},
    identifierstyle=\color{blue},
    stringstyle=\color{orange},
}
\lstset{escapechar=@,style=customc}

\begin{document}

    % Introduce a new counter for counting the nodes needed for circling
    \newcounter{nodecount}
    % Command for making a new node and naming it according to the nodecount counter
    \newcommand\tabnode[1]{\addtocounter{nodecount}{1} \tikz \node (\arabic{nodecount}) {#1};}
    
    % Some options common to all the nodes and paths
    \tikzstyle{every picture}+=[remember picture,baseline]
    \tikzstyle{every node}+=[inner sep=0pt,anchor=base]
    \tikzstyle{every path}+=[thick, rounded corners]
    
    % for tikz pict

    \maketitle
	In order  for you to work remotely, we changed this year project into a game. The game simulates a pandemic situation, very much like the one we are seeing right now. As you have seen, real-time response is a critical component of handling a pandemic efficiently. 

It is by no means our intention to take the current epidemic lightly. However, we thought that this project assignment would allow to lighten the mood a little, as everyone's morale will also be taking a serious hit during this period of confinement. Keep your spirits high and make sure you give your courses your best (all of them, not just this one). We will need talented and bright engineers in the aftermath of this crisis.  

To play this game, you don't need any equipment other than the PSOC itself.
The game is entirely coded inside one FreeRTOS task (\textit{gameTask}).
It sends "events" and we ask you to make a code that will automatically respond to those "events".\\
Depending on your response time you will increase or not some variables.

\subsection*{Useful documentation:}
\begin{itemize}
    \item Official \rtos~ documentation: \url{https://www.freertos.org/Documentation/RTOS_book.html}
    \item Getting Started with PSoC 5LP: \url{https://www.cypress.com/file/41436/download}
    \item Video example on how to use the PSoC: \url{https://www.cypress.com/video-library/PSoC}
    \item The extension board schematics: \texttt{Extension\_PSoC.pdf}
		\item Getting started with the Logic Analyzer: \url{https://learn.sparkfun.com/tutorials/using-the-usb-logic-analyzer-with-sigrok-pulseview}
\end{itemize}


    \section*{Game description}

In this game you will try to beat the current pandemic propagation by researching for a vaccine, producing medicine and containing sudden contamination burst.
The only objective is to find a cure before the population get totally infected.
\begin{itemize}
	\item The percentage of healthy population, \kw{populationCntr}, starts at $ 100\% $. \textbf{You lose} when it reaches $ 0\% $.
	\item The vaccine counter, \kw{vaccineCntr}, starts at $ 0\% $. \textbf{You win} if it reaches $ 100\% $ before the whole population is infected.
	\item The amount of medicine pills available (in million), \kw{medicineCntr}, will help you to slow down the disease propagation.
\end{itemize}

You have at your disposal one research lab.
It can either work on a cure or produce medicine.
It cannot do both at the same time and when it starts a job it should work until completion\footnote{The lab is a shared resource.}.


\subsection*{There are three types of event}

\paragraph{Vaccine Research}
Every $ 3s $, the \textit{gameTask} will drop a \textit{clue}.
If you manage to make the lab work on the \textit{clue} and return its result within $ 3s $ then the \kw{vaccineCntr} is increased by $ 3 $.

\paragraph{Spread of the disease}
Every $ 5s $, the percentage of healthy population is decreased by $ 5 - $ \kw{medicineCntr}.
The pills can only be used once.
This is why you should keep producing them.
When it is not working on a \textit{clue} the research lab should produce as many pills as possible.

\paragraph{Contamination}
The \textit{gameTask} will randomly contaminate an individual.
If you do not \textit{quarantine} the population within $ 10ms $, $ 20\% $ get infected instantly.

\section*{Specifications}
Open the project \kw{pandemic}.

The whole game is coded in \textit{pandemic.c}.
Read its documentation in the header file (\textit{pandemic.h}) to better understand how to use it. You can call the following functions from the \textit{pandemic.h} library: 
\begin{itemize}
	\item \texttt{quarantine( void )}
	\item \texttt{assignMissionToLab( Token mission )}
	\item \texttt{shipMedicine( Token medicine )}
	\item \texttt{shipVaccine( Token vaccine )}
	\item \texttt{getPopulationCntr( void )}
	\item \texttt{getVaccineCntr( void )}
	\item \texttt{getMedicineCntr( void )}
\end{itemize}
The functions are self-explanatory, and the comments in the (\textit{pandemic.h}) file should help you understand how to use them. 
\\

You should only write your code in \textit{main.c}.
You may use as many task as you want as long as their priority is lower than the \textit{gameTask} (i.e. \textit{gameTask} has a priority of 20, so your tasks cannot have a priority higher than 19).\\

Write a task that print the \kw{populationCntr}, \kw{vaccineCntr} and \kw{medicineCntr} on the left side of the display.
The last 8 characters of the LCD are reserved to the \kw{gameTask}.\\

When the \textit{gameTask} sends a contamination it calls the \kw{releaseContamination()} function.
You should override this function in \textit{main.c} in order \textbf{to communicate with your own tasks}.
%This function should be treated as an interruption, its code should be as short as possible.
The \kw{quarantine()} function should be launched from your own task.
\textbf{The following is strictly forbidden} (otherwise it would make things too easy):
\begin{lstlisting}[
	label={lst:listing 2}
]
	void releaseContamination(){
	    quarantine();
	}
\end{lstlisting}


The \textit{gameTask} releases clues for the vaccine in a similar manner, (see \kw{releaseClue(clue)}). You should again override this function in the \textit{main.c} in order \textbf{to communicate with your own tasks}. 
When this happens your code should call the functions \kw{assignMissionToLab()} and \kw{shipVaccine()} from within your own tasks.
\begin{lstlisting}[
	label={lst:listing 2}
]
	// [...]
	vaccine = assignMissionToLab(clue);
	shipVaccine(vaccine);
	// [...]
\end{lstlisting}

To produce medicine pill use:
\begin{lstlisting}[
	label={lst:listing 2}
]
	// [...]
	medecine = assignMissionToLab(0);
	shipMedicine(medecine);
	// [...]
\end{lstlisting}



    \section*{Guidelines}

During the labs, you have seen how to handle inter-task communication, resource sharing, synchronisation, ...
Use this knowledge in a clever way.\\

We ask you to work remotely, you don't need physical access to the device to contribute.
We advise you to use Microsoft Teams as you all already have an account and you can use the Remote Desktop Control feature to compile and run code from a distance.
If for some reason that doesn't suit you, keep in mind that you only need to share the \textit{main.c} file to code.
A simple Pastebin\footnote{\url{https://pastebin.com/}} might do the trick. \\

Start by making a list of all the task that you need. Then use timing diagrams, block diagrams, state machines or other operating/functional schemes to share your ideas with your team. You should also use the logic analyzer to understand how the RTOS schedules tasks in your code. \\


\section*{Deliverables}

You need to submit the following elements to us through the UV: 
\begin{itemize}
	\item the \texttt{main.c} file  of your project (only this file). We will be checking this, so make sure you place some comments in your code!  
	\item a picture of your PSoC with the \texttt{populationCntr} at the end of the game; 
	\item A small report (maximum five pages), containing the design of your code, as well as a few traces obtained with your logic analyzer that illustrate what happens at critical events in your code (for example when the game releases a clue, a random contamination, etc.) Comment those logic analyzer traces so that we can understand what happens easily (e.g. specify the state that each task is in at each point in time). 
\end{itemize}
Join all of these into a single .zip file before uploading on the UV. 

\begin{flushright}
\section*{Good luck!}
\end{flushright}


    \vfill
    \footnotesize{
        Found an error? Let us know: \url{https://github.com/BEAMS-EE/ELECH410/issues}
    }

\end{document}
