\section*{Game description}

In this game you will try to beat the current pandemic propagation by researching for a vaccine, producing medicine and containing sudden contamination burst.
The only objective is to find a cure before the population get totally infected.
\begin{itemize}
	\item The percentage of healthy population, \kw{populationCntr}, starts at $ 100\% $. \textbf{You lose} when it reaches $ 0\% $.
	\item The vaccine counter, \kw{vaccineCntr}, starts at $ 0\% $. \textbf{You win} if it reaches $ 100\% $ before the whole population is infected.
	\item The amount of medicine pills available (in million), \kw{medicineCntr}, will help you to slow down the disease propagation.
\end{itemize}

You have at your disposal one research lab.
It can either work on a cure or produce medicine.
It cannot do both at the same time and when it starts a job it should work until completion\footnote{The lab is a shared resource.}.


\subsection*{There are three types of event}

\paragraph{Vaccine Research}
Every $ 3s $, the \textit{gameTask} will drop a \textit{clue}.
If you manage to make the lab work on the \textit{clue} and return its result within $ 3s $ then the \kw{vaccineCntr} is increased by $ 3 $.

\paragraph{Spread of the disease}
Every $ 5s $, the percentage of healthy population is decreased by $ 5 - $ \kw{medicineCntr}.
The pills can only be used once.
This is why you should keep producing them.
When it is not working on a \textit{clue} the research lab should produce as many pills as possible.

\paragraph{Contamination}
The \textit{gameTask} will randomly contaminate an individual.
If you do not \textit{quarantine} the population within $ 10ms $, $ 20\% $ get infected instantly.

\section*{Specifications}
Open the project \kw{pandemic}.

The whole game is coded in \textit{pandemic.c}.
Read its documentation in the header file (\textit{pandemic.h}) to better understand how to use it. You can call the following functions from the \textit{pandemic.h} library: 
\begin{itemize}
	\item \texttt{quarantine( void )}
	\item \texttt{assignMissionToLab( Token mission )}
	\item \texttt{shipMedicine( Token medicine )}
	\item \texttt{shipVaccine( Token vaccine )}
	\item \texttt{getPopulationCntr( void )}
	\item \texttt{getVaccineCntr( void )}
	\item \texttt{getMedicineCntr( void )}
\end{itemize}
The functions are self-explanatory, and the comments in the (\textit{pandemic.h}) file should help you understand how to use them. 
\\

You should only write your code in \textit{main.c}.
You may use as many task as you want as long as their priority is lower than the \textit{gameTask} (i.e. \textit{gameTask} has a priority of 20, so your tasks cannot have a priority higher than 19).\\

Write a task that print the \kw{populationCntr}, \kw{vaccineCntr} and \kw{medicineCntr} on the left side of the display.
The last 8 characters of the LCD are reserved to the \kw{gameTask}.\\

When the \textit{gameTask} sends a contamination it calls the \kw{releaseContamination()} function.
You should override this function in \textit{main.c} in order \textbf{to communicate with your own tasks}.
%This function should be treated as an interruption, its code should be as short as possible.
The \kw{quarantine()} function should be launched from your own task.
\textbf{The following is strictly forbidden} (otherwise it would make things too easy):
\begin{lstlisting}[
	label={lst:listing 2}
]
	void releaseContamination(){
	    quarantine();
	}
\end{lstlisting}


The \textit{gameTask} releases clues for the vaccine in a similar manner, (see \kw{releaseClue(clue)}). You should again override this function in the \textit{main.c} in order \textbf{to communicate with your own tasks}. 
When this happens your code should call the functions \kw{assignMissionToLab()} and \kw{shipVaccine()} from within your own tasks.
\begin{lstlisting}[
	label={lst:listing 2}
]
	// [...]
	vaccine = assignMissionToLab(clue);
	shipVaccine(vaccine);
	// [...]
\end{lstlisting}

To produce medicine pill use:
\begin{lstlisting}[
	label={lst:listing 2}
]
	// [...]
	medecine = assignMissionToLab(0);
	shipMedicine(medecine);
	// [...]
\end{lstlisting}

