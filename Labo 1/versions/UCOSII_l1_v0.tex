\documentclass[10pt,a4paper]{article}
\usepackage[utf8x]{inputenc}
\usepackage[OT2,T1]{fontenc}
%\usepackage{stringenc} % for grffile
\usepackage{ucs}
\usepackage{amsthm} %numéroter les questions
\usepackage[english]{babel}
\usepackage{datetime}
\usepackage{xspace} % typographie IN
\usepackage{hyperref}% hyperliens
\usepackage[all]{hypcap} %lien pointe en haut des figures
\usepackage[english]{varioref} %voir x p y
\usepackage{fancyhdr}% en têtes
%\input cyracc.def
\usepackage[]{graphicx} %include pictures
%\usepackage[encoding,inputencoding=utf8,filenameencoding=utf8]{grffile}
%\usepackage[extendedchars,inputencoding=latin1,filenameencoding=latin1]{grffile}
\usepackage[siunitx ]{circuitikz}
\usepackage{gnuplottex}
\usepackage{ifthen}
\graphicspath{{./figures/}}
%\usepackage{array}
\usepackage{amsmath}
\usepackage[]{xcolor}
\usepackage{tikz}
\usepackage{tikz-timing}
\usetikzlibrary{scopes}
\usetikzlibrary{backgrounds}
\usepackage{listings}
\usepackage{enumitem}
\usepackage[top=1 in, bottom=1 in, left=1.3 in, right=1 in]{geometry} % Yeah, that's bad to play with margins
\usepackage[]{pdfpages}
\usepackage{pdflscape}
\usepackage[]{attachfile}
%\usepackage{colortbl}
%\usepackage{multirow}
\usepackage{booktabs}
\usepackage{makecell}
\usepackage[ ]{subfig}
%\usepackage{rotating}
\usepackage{upgreek}

\newdateformat{mydate}{2013--2014}%hack pour remplacer \THEYEAR

%cyr
%\newcommand\textcyr[1]{{\fontencoding{OT2}\fontfamily{wncyr}\selectfont #1}}


\newboolean{corrige}
%\setboolean{corrige}{true}%corrigé
\setboolean{corrige}{false}% pas de corrigé

\newboolean{annexes}
%\setboolean{annexes}{true}%annexes
\setboolean{annexes}{false}% pas de annexes

\newboolean{mos}
%\setboolean{mos}{true}%annexes
\setboolean{mos}{false}% pas de annexes

\usepackage{aeguill} %guillemets

%% fancy header & foot
\pagestyle{fancy}
\lhead{[ELEC-H-410] Real-Time systems Labo n° 1: \uCOSII}
\rhead{\mydate\today\\ page \thepage}
\chead{\ifthenelse{\boolean{corrige}}{Corrigé}{}}
\cfoot{}
%%

\pdfinfo{
/Author (Yannick Allard - Raoul Sommeillier, ULB -- BEAMS)
/Title (Labo n° 3 ELEC-H-410, uCOS-II)
/ModDate (D:\pdfdate)
}

\hypersetup{
pdftitle={Labo n° 1 [ELEC-H-410] Real-Time systems},
pdfauthor={Yannick Allard - Raoul Sommeillier, ©2014 ULB - BEAMS  },
pdfsubject={uCOS-II}
}

\theoremstyle{definition}% questions pas en italique
\newtheorem{E}{Exercise}[] % numéroter les questions [section] ou non []

\newcommand{\reponse}[1]{% pour intégrer une réponse : \reponse{texte} : sera inclus si \boolean{corrige}
	\ifthenelse {\boolean{corrige}} {\paragraph{Réponse :} #1} {}
 }

\newcommand{\addcontentslinenono}[4]{\addtocontents{#1}{\protect\contentsline{#2}{#3}{#4}{}}}

\newcommand{\on}[1]{\operatorname{#1}}

\newcommand{\reg}[1]{\texttt{reg#1}}

\newcommand{\uCOSII}{$\upmu$C/OS-II }

\newcommand{\kw}[1]{\texttt{#1}}

\def\labelitemi{--}
\setlist{parsep=0pt,itemsep=0pt,style=standard,leftmargin=\parindent, align=left} % pas d'espace prohibitif entre les items
\setlist{nolistsep}

\newcolumntype{C}[1]{>{\centering\let\newline\\\arraybackslash\hspace{0pt}}m{#1}}

%\setlength{\tabcolsep}{0pt} %no extra space in cells to keep constant tabular width

\date{\vspace{-1cm}\mydate\today}
\title{\vspace{-2cm} Labo n° 1\\ Real-Time systems [ELEC-H-410]\\ Realization of an application under \uCOSII \ifthenelse{\boolean{corrige}}{~\\Corrigé}{}}

%\author{\vspace{-1cm}}%\textsc{Yannick Allard}}


\lstdefinestyle{customasm}{
 % belowcaptionskip=1\baselineskip,
 % frame=L,
 % xleftmargin=\parindent,
  language=[x86masm]Assembler,
  basicstyle=\footnotesize\ttfamily,
  commentstyle=\itshape\color{purple!40!black},
  comment=[l]//,
}
\lstdefinestyle{customc}{
  belowcaptionskip=1\baselineskip,
  breaklines=true,
  frame=L,
  xleftmargin=\parindent,
  language=C,
  showstringspaces=false,
  basicstyle=\footnotesize\ttfamily,
  keywordstyle=\bfseries\color{green!40!black},
  commentstyle=\itshape\color{purple!40!black},
  identifierstyle=\color{blue},
  stringstyle=\color{orange},
}
\lstset{escapechar=@,style=customc}

\begin{document}

% Introduce a new counter for counting the nodes needed for circling
\newcounter{nodecount}
% Command for making a new node and naming it according to the nodecount counter
\newcommand\tabnode[1]{\addtocounter{nodecount}{1} \tikz \node (\arabic{nodecount}) {#1};}

% Some options common to all the nodes and paths
\tikzstyle{every picture}+=[remember picture,baseline]
\tikzstyle{every node}+=[inner sep=0pt,anchor=base]
\tikzstyle{every path}+=[thick, rounded corners]

% for tikz pict


\maketitle
\section*{Purpose}
During the 5 following laboratories you will carry out a project having for goal to design a distributed
alarm; this project will enable you to study:
\begin{itemize}
\item how to program under a real-time OS: \uCOSII;
\item properties and uses of the network CAN;
\end{itemize}

The laboratories will be divided into two parts:
\begin{itemize}
\item the first three labs will have as goals to familiarize you with the programming under \uCOSII and
the use of the CAN network.
\item  the two following labs will be used for the realization of the distributed alarm, using the concepts
previously acquired.
\end{itemize}
 
\noindent
Useful documents are stored on the lab server:\\
 \verb!\ELEC-H-410\Useful Documents\uCOSII_RefMan.PDF!\marginpar{OK update this}

\section{First lab}
During this first lab, you will learn how to write a task under \uCOSII, to make it periodic and to assign
it a priority, in an intelligent way. The hardware will be composed of a microcontroller board and a logic
analyser.

If you are not confident with C programming, read XXX-XXX.

Principles of the logic analyser are explained in the chapter 9; an how to guide for the Asix Sigma2 logic analyser is in Appendix\ref{ap:la}.

\subsection{Creation of a task under \uCOSII}

A task is a succession of instructions doing a specific operation. Contrary to a function, a task cannot
return a value. Moreover you do not have any direct influence on the order of execution of the various
tasks which you create. Indeed, it is the operating system which is given the responsibility to schedule
the tasks and thus to choose which task must be carried out at which time on the processor. \uCOSII 
is a premptive RTOS based on fixed priorities that you have assigned to the tasks. The choice of those
priorities is thus crucial so that the system behaves as you wish. This is why the second part of this lab
will be related to the judicious choice of the priorities.

First, you will learn how to create a single simple task in \uCOSII and to initiate the execution of the
operating system.

Using the Integrated Development Environment MPLab, open the project \textbf{``Example1"} in the folder
\verb!\ELEC-H-410\uCOS-II\Exercices.!\marginpar{OK update that}

In the file main.c you will find the function main (see Listing \ref{lst:listing 1}) in which are executed :
\begin{itemize}
\item the initialization of \uCOSII and all its internal variables : \kw{OSInit()}
\item the creation of the task \kw{AppTaskStart}: \kw{OSTaskCreateExt()}
\item the starting of \uCOSII: \kw{OSStart()}
\end{itemize}

This structure cannot vary. The operating system must indeed be initialized before any creation of task
and at least one task must have been created before giving control to OS. If no task were present in
the system when the \kw{OSStart()} function is called, \uCOSII would launch a useless task "Idle" and do
nothing else would be executed by the CPU.

For more details on the parameters sent during the creation of the task, refer to the \uCOSII user's
manual (page 113).
\newpage
\begin{lstlisting}[caption={Function main.c}, label={lst:listing 1}]
#include <includes.h>
CPU_INT16S main (void)
{
CPU_INT08U err;
OSInit();
 // Initialize "uC/OS-II "
OSTaskCreateExt(
AppTaskStart, // creates AppStartTask
	(void *)0,
	(OS_STK *)&AppTaskStartStk[0],
	APP_TASK_STARTPRIO,
	APP_TASK_START_PRIO,
	(OS_STK *)&AppTaskStartStk[APP_TASK_START_STK_SIZE-1],
	APP_TASK_START_STK_SIZE,
	(void *)0,
	OS_TASK_OPT_STK_CHK | OS_TASK_OPT_STK_CLR);
OSStart();
 // Start multitasking (i.e. give control to uC/OS-II)
}
return (-1);
// Return an error - This line of code is unreachable

\end{lstlisting}
\marginpar{OK add caption and refr to listing}

\subsection{How to write a task}
\begin{itemize}
\item the task must be written like a function which returns nothing (\kw{void})
\item the task must contain an infinite loop : use one of the 2 structures \kw{for( ; ; ){\dots}} or \kw{while(1){\dots}}
\item a task must always call at least one of the services of \uCOSII that will make the task “waiting”
like \kw{OSTimeDly}, \kw{OSTaskSuspend()}, \kw{OSSemPend()}, \kw{OSMailboxPend()}, \kw{OSMutexPend()}. Since
\uCOSII is preemptive the currently running task has got the highest priority among all “ready”
tasks, hence if no event occurs (like an ISR making a higher priority task ready or the current task
giving the control back to the scheduler) no other tasks will ever run.

\end{itemize}

\begin{lstlisting}[caption={task1.c}, label={lst:listing 2}]
void task1 (void *data){
...
}
for(;;){
...
OSTimeDly(10);
}
//ask the RTOS to put task1 in "waiting"
//state for at least 9 ticks

\end{lstlisting}

\subsection{Put a task to sleep for some time}
Sometimes, it is necessary to let a task sleep for a while (maybe the job is complete...) \marginpar{reformuler}

Let a task sleep in “waiting” state for some time \kw{OSTimeDly(INT16U tick\_nbr)};
The parameter \kw{tick\_nbr\_tick} is an unsigned 16bit integer (ranging from 0 and 65535) which
determines the number of ticks during which the task will sleep. The timer creating the periodic
interrupts has been configured for a frequency of 1kHz, hence 1 tick = 1 ms.
More precisely the task will sleep at least (\kw{tick\_nmbr-1}), if you want to be sure to sleep during 1 tick
you should specifiy \kw{tick\_nmbr=2}.
To demonstrate that, draw a chronogram of tick interrupts and imagine where the call \kw{OSTimeDly()}
could occur.

\E{
Create one second task in the Example1 project which lights a LED of the uC board at a frequency of
1Hz. 

To change the state of the LED, toggle pin \kw{LATAbits.LATA3}.
Remember that you have to configure the pin in the output direction by the instruction
\kw{TRISAbits.TRISA3 = 0}. (see \begin{verbatim} \ELEC-H-410\Useful Documents\La Carte Explorer 16.pdf p3 or \ELEC-H-410\Useful Documents\dsPIC33 Data Sheet_70286C.pdf \end{verbatim} p157)


Use the code for \kw{AppTask1} as a model.

}{}

\subsection{Creation of periodic tasks}
In Exercise1, you have created a periodic task, i.e. a task executing forever at regular intervals. In most
industrial applications, those tasks are frequent and a the periodicity should be realized with a good
precision (see example of PI controller in chapter 3 of the course).

\E{
Open the project entitled (Example\_Periodicity).
You will find 4 tasks in this example:
\begin{itemize}
\item \kw{AppTaskStart} whose only goal is to create the three other tasks
\item \kw{AppTask1} who should have a period of 10ms;
\item \kw{AppTask2} who should have a period of 50ms;
\item \kw{AppTask3} who should have a period of 100ms.
\end{itemize}

\begin{itemize}
\item Switch the logical analyser on and launch the display interface on the PC.
\item Open the configuration file elec-h-410.lwc in the folder /MesDocuments/ELEC-H-410.\marginpar{change that}
\item Start your program Example\_Periodicity on the microcontroller and launch a first data acquisition.
with the logic analyser.
\item Observe the evolution of the value of the bus \kw{RunningTaskId} which shows the identifier of the tasks running on the processor. Observe preemptions of certain tasks when a higher priority task is active (see signals \kw{Task1Active}, \kw{Task2Active} et \kw{Task3Active} whose value is 1 when the tasks \kw{AppTask1}, \kw{AppTask2} and \kw{AppTask3} are respectively active, \textit{i.e.}, between its first instruction until its completion).
\item Use the logic analyser to measure the real period of real activation of each task. Are they exactly in conformity with the desired periods? Identify 2 causes of these errors.
\end{itemize}
}{}

\subsubsection{Use of \kw{OSTimeGet()}}
\uCOSII provides the \kw{OSTimeGet()} function which returns a 32 bit integer (\kw{INT32U}) representing the
number of ticks since the launching of OS.

\E{Compute after how long this counter will overflow.}{}

\E{Use \kw{OSTimeGet()} in each task to compensate for the error over the period.}{}

\subsubsection{Use of a software timer}

It is possible to use software timers in \uCOSII. Those are used exactly in the same way as hardware
timers, except that they are entirely managed by the operating system and that they are synchronized
on the ticks of the system.
The function \kw{OSTimerCreate()} allows to create a software timer (see manual for the parameters
details) and \kw{OSTmrStart()} to start it. When a timer expires, it calls a function whose pointer was given
to it in the parameters.

Open the project (Example\_Timer).
You will find the same 4 tasks as in the previous example except that their period are generated by
using three timers softwares.

Functions \kw{OSTaskSuspend()} and \kw{OSTaskResume()} allow to suspend and restart the execution of a
specific task.

\E{Check with the logical analyser that the periods are strictly respected.}{}

This method for creating periodic task gives very precise results. However, it is rather heavy and
should therefore be used when this precision is absolutely required.

\E{Create a new a timer which switches a LED on after 5s.}{}

\subsubsection{Choice of the priorities}

As explained earlier, the choice of the priorities of the task is the only tool at our disposal to help the
operating system to choose which task must be running at which time. To be convinced of the
importance of a judicious choice of these priorities, we will look at a simple example.
\E{Open the project (Example\_Priorites).}{}
\begin{itemize}
\item The task \kw{AppTask1} should run every 1ms
\item The task \kw{AppTask2} should run every 100ms
\end{itemize}

\begin{itemize}
\item Check the behaviour of the tasks with the logic analyser.
\item Reverse the priorities of \kw{AppTask1} and \kw{AppTask2} and reverify what occurs.
\item By comparing the periods of each task and the priorities assigned, which systematic rule of
assignment can you deduce?
\item How is called this method to assign the priorities?
\item What happens when tasks have relative deadlines different from their periods?
\item Which scheduling algorithm would you use if you could assign priorities directly to jobs instead of
tasks?
\end{itemize}

\appendix
\section{The \textit{Asix Sigma2} logic analyser}
The Asix Sigma2 logic analyser is like:\\ 
\label{ap:la}
\begin{center}
\includegraphics[width=5cm]{sigma2_720x515.jpg}
\end{center}
\subsection{Electrical connections to the Explorer 16 board}
Connect the analyser to the extension board with the numbered ribbon cable following this scheme:
\begin{center}
\includegraphics[width=12cm]{elec_con-crop.pdf}
\end{center}
\subsection{Software on the computer}

Add some screenshots here

\subsection{Basic measurements}
There is a cursor showing the time and values of signals in the main window. To place a marker, press space. If you move your cursor, the difference between the marker and the cursor will show in a tooltip.

\end{document}